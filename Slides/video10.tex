\documentclass[xcolor=table]{beamer}

\usepackage{booktabs}
\usepackage{hyperref}
\usepackage[table]{xcolor}
\usepackage{tikz}
\usepackage{graphics}
\usetikzlibrary{calc}

\setbeamertemplate{navigation symbols}{}%remove navigation symbols

\title{Infinitely Repeated Games}
\subtitle{Game Theory}
\author{Vincent Knight}
\date{}

\begin{document}

\frame{\titlepage}

\frame{
$$
\Huge{
\begin{pmatrix}
(2,2)&(0,3)\\
(3,0)&(1,1)
\end{pmatrix}}
$$
}

\frame{
$$
\begin{pmatrix}
(2,2)&(0,3)\\
(3,0)&(1,1)
\end{pmatrix}
$$
\begin{itemize}
\item $s_C$: cooperate at every stage
\item $s_D$: defect at every stage
\end{itemize}
\onslide<2->{
$$u_1(s_C,s_C)=\sum_{i=1}^\infty\only<3->{{\color{red}{\delta^i}}}2\only<3->{<}\only<2>{>}\infty\only<3->{{\color{red}{\text{ if }|\delta|<1}}}$$
}
\onslide<4->{
\begin{center}
\framebox{Possible interpretation of $\delta$: probability of game ending at any stage.}
\end{center}
}
}

\frame{
$$\Huge \bar T = \frac{1}{1-\delta}$$
$$\Huge \frac{1}{\bar T}U_i(r,s) = ({1-\delta})U_i(r,s)$$
}

\frame{
\tikzstyle{end} = [circle, minimum width=3pt, fill, inner sep=0pt, right]
\begin{center}
\begin{tikzpicture}
    \draw (0,0) -- (4,0) node [below] {$u_1$};
    \draw (0,0) -- (0,4) node [left] {$u_2$};
    \draw (3,0) node[end] (A) {} node [above=.3cm,right] {$(3,0)$};
    \draw (0,3) node[end] (B) {} node [above=.3cm,right] {$(0,3)$};
    \draw (1,1) node[end] (C) {} node [below,left] {$(1,1)$};
    \draw (2,2) node[end] (D) {} node [above=.3cm,right] {$(2,2)$};
    \draw (A) -- (D);
    \draw (D) -- (B);
    \draw (B) -- (C);
    \draw (C) -- (A);
    \draw [->] (2,3.5) node[above] {\tiny{Feasible average payoffs}} -- (.75,2);
    \pause
    \draw [dashed,thick] (C) -- ++(0,1.5);
    \draw [dashed,thick] (C) -- ++(1.5,0);
    \pause
    \draw [->] (4,1.5) node[above] {\tiny{Individually rational payoffs}} -- (1.5,1.25);
\end{tikzpicture}
\end{center}
}

\frame{
\begin{center}
\color{blue}{Folk Theorem}
\end{center}

Let $(u_1^*,u_2^*)$ be a pair of Nash equilibrium payoffs for a stage game. For every individually rational pair $(v_1,v_2)$ there exists $\bar \delta$ such that for all $1>\delta>\bar \delta>0$ there is a subgame perfect Nash equilibrium with payoffs $(v_1,v_2)$.
}

\end{document}
